\documentclass[10pt]{article}                                  %%%
\addtolength{\oddsidemargin}{-6em}                               %%%
\addtolength{\textwidth}{12em}                                   %%%
\addtolength{\topmargin}{-17ex}                                   %%%
\addtolength{\textheight}{32ex}
\font\TitleFont=cmbx10 at 24.88truept
%\font\Bfour=cmbx10 at 20.74truept
\font\DateFont=cmbx10 at 17.28truept
\font\SubTitleFont=cmdunh10 at 14.40truept
%\font\TitleFont=cmbx10 at 20.74truept
%\font\DateFont=cmbx10 at 14.40truept
%\font\SubTitleFont=cmdunh10 at 12pt
\newcommand{\st}{^{\textstyle{\ast}}}
\newcommand{\iinfo}[2]{{\raggedright{\em#1}}\\ &{\raggedright{\hspace*{.2truein}#2}}\\[1ex]}
\newcommand{\cinfo}[2]{{\raggedright{\hspace*{.2truein}\em#1}}\\ &{\raggedright{\hspace*{.2truein}\hspace*{.2truein}#2}}\\[1ex]}
\newcommand{\CONT}{{\raggedright{\bf Session for contributed papers}}\\}
\newcommand{\CONTP}{{\raggedright{\bf Parallel Sessions for contributed papers}}\\}
\newcommand{\rmB}{\ \ \ \ Room: 252&}
\newcommand{\rmA}{\ \ \ \ Room: 274&}
\newcommand{\rmC}{\ \ \ \ Room: 274&}
\newcommand{\rmD}{\ \ \ \ Room: TBA&}
% \newcommand{\rmAb}{\ \ \ \ BS 3009&}
% \newcommand{\rmBb}{\ \ \ \ BS 3012&}
% \newcommand{\rmCb}{\ \ \ \ BS 3013&}
% \newcommand{\rmAs}{\ \ \ \ LD 010&}
% \newcommand{\rmBs}{\ \ \ \ LD 004&}
% \newcommand{\rmCs}{\ \ \ \ LD 014&}
\pagestyle{empty}                                               %%%
\title{Notice}                                               %%%
\author{Marius Dadarlat}                                          %%%
\begin{document}
%\begin{flushright}{\small Final Announcement}\end{flushright}
\vspace{2ex}
\begin{center}
%\bigskip
{\TitleFont WABASH}\\
\smallskip
{\SubTitleFont EXTRAMURAL MODERN ANALYSIS}\\
\smallskip
{\TitleFont MINICONFERENCE}\\
\vspace{.2in}
 {\DateFont   October 2 and 3, 2010}\\
\vspace{.2in}
{\TitleFont Program}\\[.2in]

{\em Times given are Eastern Daylight Time}\\[.3in]

%{\em Times given are Central Daylight Time, current time for
%Central Indiana
    %and Illinois.}\\[.3in]

Talks and Registration will be in Informatics and Communications Technology Complex at IUPUI.

 The
lectures will take place in Room 252 and the contributed talks in Rooms 252/274.
\end{center} \smallskip

\noindent
{\bf Saturday:}\\[1ex]
\hspace*{.3in}{ \begin{tabular}{lp{5.3in}}
8:45&{\raggedright{Registration, Refreshments}}\\
%9:20--9:30&{\raggedright{Welcome to IUPUI}}\\
9:30--10:20&\iinfo{JOHN McCARTHY, Washington University }
{Operator Monotone functions of several variables
}
&{\raggedright{Break}}\\[2ex]
10:30--11:20&
\iinfo {DAVID KERR, Texas A\&M University}{Entropy and the variational principle for actions of sofic groups }
&{\raggedright{Break}}\\[2ex]
11:35--12:00&\CONTP \rmB\cinfo{TIMUR OIKBERG,  University of California at Irvine} {Automatic continuity of orthogonality preserving linear maps}

11:35--12:00&\CONTP \rmC\cinfo{GABRIEL PRAJITURA, SUNY-Brockport} {Criteria for irregularity }
&{\raggedright{ Lunch:  12:30 at BISTRO}}\\[2ex]
2:00--2:50&\iinfo{VERN PAULSEN,  University of Houston}{ Tensor Products of Operator Systems and the Kirchberg Conjecture }
&{\raggedright{Break}}\\[2ex]
3:05--3:55&\iinfo {ALEXANDER DRANISHNIKOV,  University of Florida}{Macroscopic dimension and essential manifolds }
&{\raggedright{Break}}\\[2ex]


\end{tabular}} 
{ \begin{tabular}{lp{5.3in}}

4:10--4:35&\CONTP \rmB\cinfo{            
JOS\'E R. CARRI\'ON, Purdue University  }{On the classification problem for a class of crossed product $C^*$-algebras
}

4:10--4:35&\CONTP \rmC\cinfo{RUHAN ZHAO, SUNY-Brockport}
 {Carleson measures, Riemann-Stieltjes and multiplication operators on $F(p,q,s)$ spaces}


4:45--5:10&\CONTP \rmB\cinfo{TAYLOR HINES,  Purdue University }{
Nontrivially Noetherian and Artinian C*-algebras}

4:45--5:10&\CONTP \rmC\cinfo{MATT MCBRIDE, IUPUI }
 {D-bar Operators on Quantum Domains}


5:20--5:45&\CONTP \rmB\cinfo{STEVE AVSEC, University of Illinois at Urbana-Champaign }{"Good" semigroups and their applications to fourier multipliers}

%5:20--5:45&\CONTP \rmC\cinfo{MATT MCBRIDE, IUPUI }
% {D-bar Operators on Quantum Domains}
 

\end{tabular}

%\noindent{\bf Saturday (Continued):}\\[1ex]
%{ \begin{tabular}{lp{5.3in}} 4:10--4:30&\CONT \rmB\cinfo{MARCUS CARLSSON,
%University of Lund} {Boundary behavior in Hilbert spaces of vector-valued
%analytic functions} \rmC\cinfo{DANA CLAHANE, University of California,
%Riverside}{Compact weighted composition operators over convex domains}\\[1ex]
% 4:40-5:00&\CONT \rmB\cinfo{TIMUR OIKHBERG, University of
%California - Irvine} {Spectrum of a compact weighted composition operator on
%the
%              Hardy space}
%   \end{tabular}
%              %\rmC\cinfo{JOANNE
%%DOMBROWSKI,
%%Wright State Univ.}{Spectral Properties of Unbounded Jacobi Operators}\\[1ex]
%%
%%5:10--5:30&\CONT \rmB\cinfo{CHRISTOPHER SEATON, Rhodes College}{
%%The Orbifold Euler Class} \rmC\cinfo{DANA CLAHANE,
%% Fullerton College}{Bounded and compact composition operators between different
%%generalized Bloch and Lipschitz spaces of the polydisk}\\[1ex]
%%
%%
%%5:40--6:00&\CONT \rmB\cinfo{PETER LOEB, Univ. of Illinois at
%%Urbana-Champaign} {Local Maximal Function Simplifies Measure
%%Differentiation} \rmC\cinfo{MONICA ILIE, Texas A$\&$M University}{
%%On Fourier algebra homomorphisms}\\[1ex]
%% \end{tabular}
%%\vskip 8pt
%
%%5:50--6:10&\CONT \rmB\cinfo{GAJATH GUNATILLAKE, Purdue University}
%%{Spectrum of a compact weighted composition operator on the
%%              Hardy space}
%
%%\noindent{ 8:00 pm \bf Social Gathering} at CARL COWEN's house\\[1ex]
\vskip 1cm
\noindent{\bf Sunday:}\\[1ex]
\hspace*{.3in}{ \begin{tabular}{lp{5.3in}}
8:30&{\raggedright{Refreshments}}\\
9:00--9:50&{\raggedright{\em DEGUANG HAN,  University of Central Florida  }

{The Feichtinger conjecture for group representation frames
}}\\[2ex]
&{\raggedright{Break}}\\[2ex]

10:00--10:50&{\raggedright{\em INDIRA CHATTERJI, Ohio University }

{Subgroup distortion and bounded cohomology
}}\\[2ex]
&{\raggedright{Break}}\\[2ex]
11:00--11:50&{\raggedright{\em IONUT CHIFAN, Vanderbilt University }\\
{Von Neumann algebras with unique group measure space Cartan subalgebra}
}\end{tabular}
%\vskip 1cm
%\hspace*{.3in}{ \begin{tabular}{lp{5.3in}}
%12:00--12:25&\CONT \rmB\cinfo{PAUL SCHUPP, University of Illinois at
%Urbana-Champaign }{Why Random Groups have Strong Mostow Rigidity}
%\end{tabular}


\vskip 1cm
\begin{center}{End of Conference}
\end{center}
\end{document}
Indira Chatterji
Alexander Dranishnikov

Ionut Chifan

Deguang Han

David Kerr

John McCarthy

Vern Paulsen
Entropy and the variational principle for actions of sofic groups

Recently Lewis Bowen introduced a notion of entropy for measure-preserving
actions of a countable sofic group on a standard probability space
admitting a generating partition with finite entropy. Using an operator
algebra perspective we develop a more general approach to sofic entropy
which produces both measure and topological dynamical invariants.
We establish the variational principle in this context and use it to compute the topological entropy of certain algebraic actions of
residually finite groups in terms of the Fuglede-Kadison determinant.
This is joint work with Hanfeng Li. 