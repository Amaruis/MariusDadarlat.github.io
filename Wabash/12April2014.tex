\documentclass[12pt]{article}
\usepackage{amsmath,amsfonts,amssymb,amsxtra,latexsym,amscd,enumerate,amsthm}
%\usepackage{amsmath, amsfonts, amssymb, amsthm, mathtools, mathrsfs}
%\documentclass[12pt]{article}
%%%%%%%%%%%%%%%
%
%    Macros that need to be changed each time!
%

\newcommand{\Date}{April 12}
\newcommand{\RideDate}{April 12}
\newcommand{\Nextdate}{TBA}
\newcommand{\BBB}{\~mdd}
%
%  Time changes Last Sunday in October and First Sunday in April!!!!!
%
\newcommand{\TimeA}{Times given  are Eastern Daylight Time,}
\newcommand{\TimeB}{which is currently local time for Central Indiana and Ohio.}
%\newcommand{\TimeA}{Times given are Central Daylight Time,}
%\newcommand{\TimeB}{which is currently local time for Central Indiana and Illinois.}
%%%%%%%%%%%%%%%%%%
\newcommand{\SpeakerOne}{CARL C.~COWEN}%>>> ALL CAPS
\newcommand{\AffilOne}{IUPUI}
\newcommand{\TitleOne}{Commutants of Finite Blaschke Product
Multiplication Operators on Bergman Spaces}
\newcommand{\ETitleOne}{Commutants of Finite Blaschke Product
Multiplication Operators on Bergman Spaces}
\newcommand{\AbstractOne}{Except in special circumstances, it is quite difficult to determine conditions 
that characterize which operators commute with a given operator.  Such special 
circumstances include self-adjoint and normal operators (where the spectral theorem 
can be used) and cases in which the operator in question has a rich point spectrum.  
The results in this latter situation often come from the application of  the easy observation
that if $A$ and $B$ commute, the eigenspaces of $A$ are invariant for $B$.   

\hspace*{1em}  If ${\mathcal H}$ is a Hilbert space of analytic functions on the 
unit disk and $T_z$ is the operator of  multiplication by $z$, it is well known that the 
commutant of $T_z$ is the collection of multiplication operators $T_f$ where 
$f$ is a bounded analytic function on the disk, $f$ is in multiplier algebra for ${\mathcal H}$, 
and $(T_fh)(z)=f(z)h(z)$.

\hspace*{1em}  In the 1970's and 80's, the question ``Which operators on the Hardy 
space $H^2({\mathbb D})$ commute with $T_f$ for $f$ a bounded analytic function 
on the disk?" was investigated.   More recently, there has been interest in 
this question for the Bergman space $A^2({\mathbb D})$ and weighted Bergman
spaces.  In this talk, an overview of the work of thirty years ago will be presented 
and we will consider this question for $f=B$, a finite Blaschke product, for $T_B$ 
acting on a broad collection of spaces containing $H^2({\mathbb D})$, a question 
that has wider consequences than  might be expected.  In particular, we show that 
the commutants of the operators $T_B$ are the same on all of these spaces!  

\hspace*{1em}  This is joint work with Rebecca G.~Wahl at Butler University.
 }
%%%%%%%%%%%%%%%%%%
\newcommand{\SpeakerTwo}{CALEB ECKHARDT}%>>> ALL CAPS
\newcommand{\AffilTwo}{ Miami University }
\newcommand{\TitleTwo}{Unitary representations of polycyclic groups}
\newcommand{\ETitleTwo}{Unitary representations of polycyclic groups}
\newcommand{\AbstractTwo}{If $\Gamma$ is a non-Type I discrete group, there is essentially no hope of a reasonable characterization of  its irreducible representations up to unitary equivalence.  If one instead focuses on only the C*-algebras generated by irreducible representations of $\Gamma$ the situation is a little more promising. Recent events in the theory of C*-algebras suggest that--in the case $\Gamma$ is a finitely generated torsion free nilpotent group--ordered K-theory may serve as a complete invariant for the C*-algebras generated by irreducible representations of $\Gamma.$ In this talk we will discuss quasidiagonality of unitary representations of polycyclic groups (all finitely generated nilpotent groups are polycyclic) and how this feeds into the larger goal of characterizing C*-algebras generated by irreducible representation of nilpotent groups. Some of this is joint work with Craig Kleski and Paul McKenney.

 }
%%%%%%%%%%%%%%%%%%%%%%%%%%%%%%%%%%%%%%%%%%%%%%%%%%%%%%%%%%%%%%%%%%%%%%%%%
%%%%%%%%%%%%%%%%%%%%%%%%%%%%%%%%%%%%%%%%%%%%%%%%%%%%%%%%%%%%%%%%%%%%%%%%%
\addtolength{\oddsidemargin}{-.8truein}
\addtolength{\textwidth}{1.6truein}  \addtolength{\topmargin}{-10ex}
\addtolength{\textheight}{30ex} \newcommand{\s}{\space}
\renewcommand{\arraystretch}{.75} \font\TitleFont=cmbx10 at 24.88truept
\font\DateFont=cmbx10 at 17.28truept \font\SubTitleFont=cmdunh10 at 14.40truept
%\font\Ding=Dingbats at 17.28truept
\pagestyle{empty}
\title{Notice}
\author{Marius Dadarlat}
\begin{document}
\vspace{5ex}
\begin{center}
\bigskip
{\TitleFont WABASH}\\
\smallskip
{\SubTitleFont EXTRAMURAL MODERN ANALYSIS}\\
\smallskip
{\TitleFont SEMINAR}\\
\vspace{.25in}
{\DateFont \Date}\\
\vspace{.25in}
{\large\bf 2:00 p.m.}\\
\medskip
{\large\bf at}\\
\medskip
{\TitleFont Wabash College}\\
\medskip
{\large\bf in rooms 114 and 118 Baxter Hall}\\
\bigskip
\bigskip
{\em \TimeA\\ \TimeB}\\
\end{center}
\bigskip
\hspace*{.5in}{\bf \begin{tabular}{lcp{5.0in}}
2:00--2:30&&{\raggedright{{\em Refreshments and conversation}}}\\[.1truein]
2:30--3:30&&{\raggedright{\TitleOne}}\\
    &&{\raggedright{{\em \SpeakerOne, \AffilOne}}}\\[.1truein]
3:30--4:00&&{\raggedright{{\em More refreshments and conversation}}}\\[.1truein]
4:00--5:00&&{\raggedright{\TitleTwo}}\\
    &&{\raggedright{{\em \SpeakerTwo, \AffilTwo}}}\\[.1truein]
5:00--...&&{\raggedright{\em Refreshments and farewells}}\\
\end{tabular}}

\begin{center}The purpose of Wabash Seminar talks is to present
  surveys of interest to all analysts,\\
including graduate students and scholars working
  in areas far from the speaker's specialty.\\
Come and meet your fellow
  analysts, learn what's going on, and spread the word.\\
\bigskip
\bigskip
\fbox{\SubTitleFont Next Meeting:\ \ \Nextdate}\\
\bigskip \bigskip
{\it For further information call}\\
\medskip
Marius Dadarlat, Purdue University, (765) 494--1940\\
E--mail: mdd@math.purdue.edu\\
Web:
http://www.math.purdue.edu/\~{\hspace{-.1em}}mdd/Wabash/\end{center}
\vfill

\newpage

\vspace{5ex}
\begin{center}{\bf \TitleOne}\\[.2in]
\SpeakerOne\\[.2in]
\begin{minipage}{5.8in}
\setlength{\parindent}{.3in}\par
\AbstractOne
\end{minipage}\\[.7in]
{\bf \TitleTwo}\\[.2in]
\SpeakerTwo\\[.2in]
\begin{minipage}{5.8in}
\setlength{\parindent}{.3in}\par
\AbstractTwo
\end{minipage}\\[.42in]
%\EMail
\end{center}
%%%%%%%%%%%%%%%%%
\typeout{                       }
\typeout{\s\s\s\s\s\s\s\s\s\s\s\s WABASH} \typeout{\s\s EXTRAMURAL
MODERN ANALYSIS} \typeout{\s\s\s\s\s\s\s\s\s\s\s\s SEMINAR}
\typeout{\s\s\s\s\s\s\s at Wabash College} \typeout{ }
\typeout{\s\s\s\s\s\s\s\s\s\s\s \Date          } \typeout{ }
\typeout{                       } \typeout{(\TimeA)}
\typeout{(\TimeB)} \typeout{ } \typeout{ } \typeout{2:30--3:30\s\s
\ETitleOne } \typeout{\s\s\s\s\s\s\s\s\s\s\s\s\s\s \SpeakerOne,
\AffilOne} \typeout{ } \typeout{4:00--5:00\s\s  \ETitleTwo }
\typeout{\s\s\s\s\s\s\s\s\s\s\s\s\s\s \SpeakerTwo, \AffilTwo}
\typeout{ } \typeout{ } \typeout{For further information call: }
\typeout{\s\s\s\s\s\s Marius Dadarlat, Purdue University, (765)
494-1940 } \typeout{\s\s\s\s\s\s\s\s E--mail: mdd@math.purdue.edu
} \typeout{\s\s\s Web:
http://www.math.purdue.edu/~mdd/Wabash/wabash.html } \typeout{ }
\typeout{Next Meeting: \Nextdate}

\typeout{                       }
\typeout{\s\s\s\s\s\s\s\s\s\s\s\s WABASH} \typeout{\s\s EXTRAMURAL
MODERN ANALYSIS} \typeout{\s\s\s\s\s\s\s\s\s\s\s\s SEMINAR}
\typeout{\s\s\s\s\s\s\s at Wabash College} \typeout{ }
\typeout{\s\s\s\s\s\s\s\s\s\s\s \Date          }  \typeout{ }
\typeout{2:30--3:30\s\s   \ETitleOne }
\typeout{\s\s\s\s\s\s\s\s\s\s\s\s\s\s \SpeakerOne, \AffilOne}
\typeout{ } \typeout{4:00--5:00\s\s  \ETitleTwo }
\typeout{\s\s\s\s\s\s\s\s\s\s\s\s\s\s \SpeakerTwo, \AffilTwo}
\typeout{ } \typeout{ }

%%%%%%%SPLIT$$$
\end{document}

\newpage
\vspace{5ex}
\begin{center}
\bigskip
{\TitleFont WABASH}\\
\smallskip
{\SubTitleFont EXTRAMURAL MODERN ANALYSIS}\\
\smallskip
{\TitleFont SEMINAR}\\
\vspace{.25in}
{\DateFont \Date}\\
\vspace{.25in}
{\large\bf 2:00 p.m.}\\
\medskip
{\large\bf at}\\
\medskip
{\TitleFont Wabash College}\\
\medskip
{\large\bf in rooms 114 and 118 Baxter Hall}\\
\bigskip
\bigskip
{\em Cars will be leaving from the Math Sciences Building at 1:30 p.m.
(sharp!).\\
(Meet near the elevators on the main floor.)\\
If you wish to ride, please tell Marius Dadarlat (Math 708; phone
41940) by Thursday,
\RideDate.}\\
\end{center}
\bigskip
\hspace*{.5in}{\bf \begin{tabular}{lcp{5.0in}}
2:00--2:30&&{\raggedright{{\em Refreshments and conversation}}}\\[.1truein]
2:30--3:30&&{\raggedright{\TitleOne}}\\
    &&{\raggedright{{\em \SpeakerOne, \AffilOne}}}\\[.1truein]
3:30--4:00&&{\raggedright{{\em More refreshments and conversation}}}\\[.1truein]
4:00--5:00&&{\raggedright{\TitleTwo}}\\
    &&{\raggedright{{\em \SpeakerTwo, \AffilTwo}}}\\[.1truein]
5:00--...&&{\raggedright{\em Refreshments and farewells}}\\
\end{tabular}}

\begin{center}The purpose of Wabash Seminar talks is to present
  surveys of interest to all analysts,\\
including graduate students and scholars working
  in areas far from the speaker's specialty.\\
Come and meet your fellow
  analysts, learn what's going on, and spread the word.\\
\bigskip
\bigskip
\fbox{\SubTitleFont Next Meeting:\ \ \Nextdate}\\
\bigskip \bigskip
{\it For further information contact}\\
\medskip
Marius Dadarlat, Math 708, 49--41940\\
E--mail: mdd@math.purdue.edu\\
Web:
http://www.math.purdue.edu/\~{\hspace{-.1em}}mdd/Wabash/wabash.html
\end{center}
\vfill

\newpage

\vspace{5ex}
\begin{center}{\bf \TitleOne}\\[.2in]
\SpeakerOne\\[.2in]
\begin{minipage}{5.8in}
\setlength{\parindent}{.3in}\par
\AbstractOne
\end{minipage}\\[.7in]
{\bf \TitleTwo}\\[.2in]
\SpeakerTwo\\[.2in]
\begin{minipage}{5.8in}
\setlength{\parindent}{.3in}\par
\AbstractTwo
\end{minipage}\\[.42in]
%\EMail
\end{center}



\typeout{                       }
\typeout{\s\s\s\s\s\s\s\s\s\s\s\s WABASH} \typeout{\s\s EXTRAMURAL
MODERN ANALYSIS} \typeout{\s\s\s\s\s\s\s\s\s\s\s\s SEMINAR}
\typeout{\s\s\s\s\s\s\s at Wabash College} \typeout{ }
\typeout{\s\s\s\s\s\s\s\s\s\s\s \Date          } \typeout{ }
\typeout{                       } \typeout{(\TimeA)}
\typeout{(\TimeB)} \typeout{ } \typeout{ } \typeout{2:30--3:30\s\s
\ETitleOne } \typeout{\s\s\s\s\s\s\s\s\s\s\s\s\s\s \SpeakerOne,
\AffilOne} \typeout{ } \typeout{4:00--5:00\s\s  \ETitleTwo }
\typeout{\s\s\s\s\s\s\s\s\s\s\s\s\s\s \SpeakerTwo, \AffilTwo}
\typeout{ } \typeout{ } \typeout{For further information call: }
\typeout{\s\s\s\s\s\s Marius Dadarlat, Purdue University, (765)
494-1940 } \typeout{\s\s\s\s\s\s\s\s E--mail: mdd@math.purdue.edu
} \typeout{\s\s\s Web:
http://www.math.purdue.edu/~mdd/Wabash/wabash.html } \typeout{ }
\typeout{Next Meeting: \Nextdate}

\typeout{                       }
\typeout{\s\s\s\s\s\s\s\s\s\s\s\s WABASH} \typeout{\s\s EXTRAMURAL
MODERN ANALYSIS} \typeout{\s\s\s\s\s\s\s\s\s\s\s\s SEMINAR}
\typeout{\s\s\s\s\s\s\s at Wabash College} \typeout{ }
\typeout{\s\s\s\s\s\s\s\s\s\s\s \Date          }  \typeout{ }
\typeout{2:30--3:30\s\s   \ETitleOne }
\typeout{\s\s\s\s\s\s\s\s\s\s\s\s\s\s \SpeakerOne, \AffilOne}
\typeout{ } \typeout{4:00--5:00\s\s  \ETitleTwo }
\typeout{\s\s\s\s\s\s\s\s\s\s\s\s\s\s \SpeakerTwo, \AffilTwo}
\typeout{ } \typeout{ }


\typeout{ Cars will be leaving from the Math Sciences Building }
\typeout{at 1:30 p.m. (sharp!) (Meet near the elevators on the
main floor.)} \typeout{If you wish to ride, please tell Marius
Dadarlat} \typeout{(Math 708; phone 41940) by Thursday,
\RideDate.}
\end{document}
