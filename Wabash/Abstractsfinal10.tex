\documentclass[11pt,leqno]{article}
\usepackage{amscd}
\usepackage{amssymb}
\usepackage{amsthm}

%\documentstyle[11pt,fleqn]{article}
\addtolength{\oddsidemargin}{-4em}
\addtolength{\textwidth}{8em} \addtolength{\topmargin}{-10ex}
\addtolength{\textheight}{30ex}
\newfont{\eu}{eufm10 scaled\magstephalf}
\font\TitleFont=cmbx10 at 20.74truept
\font\DateFont=cmbx10 at 14.40truept
\font\SubTitleFont=cmdunh10 at 12pt
\newcommand{\Title}[1]{\noindent {\bf #1}\\}
\newcommand{\Speaker}[1]{\hspace*{2em} #1,\ }
\newcommand{\Affil}[1]{\noindent{\em #1}\\}
\newcommand{\Abstract}[1]{ #1 \vspace{.3in}}
\newcommand{\st}{^{\textstyle{\ast}}}
\newcommand{\li}{\langle}
\newcommand{\ri}{\rangle}
\newcommand{\vp}{\nu}
\newcommand{\rmA}{\ \ \ \ LE 101}
\newcommand{\rmB}{\ \ \ \ LE 102}
\newcommand{\rmC}{\ \ \ \ LE 103}
\newcommand{\rmD}{\ \ \ \ LE 104}
\pagestyle{empty}
\title{Test format}
\author{Marius Dadarlat}
\begin{document}
\begin{center}
\bigskip
{\TitleFont WABASH}\\
\smallskip
{\SubTitleFont EXTRAMURAL MODERN ANALYSIS}\\
\smallskip
{\TitleFont MINICONFERENCE}\\
\vspace{.2in} {\DateFont   October 3 and 4, 2009}\\
\vspace{.2in}
{\TitleFont Abstracts}\\
\vspace{.4in}
{\DateFont Invited Talks}\\
\end{center}
\bigskip

\hspace*{-4em} 9:30--10:20, Saturday, Room: 252 \\
\Title{Operator Monotone functions of several variables} \Speaker{John McCarthy} \Affil{Washington University} \Abstract{
 In 1934 K. Lowner characterized functions that preserve operator
ordering,
i.e. real-valued functions f  with the property  that
if A and B are self-adjoint matrices and $A \leq B $,
then $f(A) \leq f(B)$.

We will discuss what functions of two variables $g$ have the property
that if $(A_1,A_2)$ is a pair of commuting self-adjoints, $(B_1,B_2)$ is
another,
and $A_1 \leq B_1 $ and $A_2 \leq B_2$, then
$g(A_1,A_2) \leq g(B_1, B_2)$.

This talk is based on joint work with Jim Agler and Nicholas Young.
.}


\hspace*{-4em}
10:30--11:20, Saturday, Room: 252 \\
\Title{Entropy and the variational principle for actions of sofic groups
 } \Speaker{David Kerr} \Affil{ Texas A\@M University } \Abstract{
Recently Lewis Bowen introduced a notion of entropy for measure-preserving
actions of a countable sofic group on a standard probability space
admitting a generating partition with finite entropy. Using an operator
algebra perspective we develop a more general approach to sofic entropy
which produces both measure and topological dynamical invariants.
We establish the variational principle in this context and use it to compute the topological entropy of certain algebraic actions of
residually finite groups in terms of the Fuglede-Kadison determinant.
This is joint work with Hanfeng Li.}


\hspace*{-4em} 2:00--2:50, Saturday, Room: 252 \\
\Title{Tensor Products of Operator Systems and the Kirchberg Conjecture} \Speaker{Vern Paulsen}
\Affil{  University of Houston} \Abstract{ This talk is based on joint work with K.H. Hoon, A. Kavruk, I.Todorov and M. Tomforde. We have begun a systematic study of tensor products of operator systems that somewhat parallels the tensor theory of operator spaces. We have identified some important tensor products and begun to study the operator systems that preserve various pairs of tensor products. This theory can be thought of as a refinement of classical nuclearity.
We prove that Kirchberg's conjecture(which is equivalent to the Connes' Embedding conjecture) is equivalent to the assertion that every (min,er)-nuclear operator system is also (el,c)-nuclear.  }
\newpage

\hspace*{-4em} 3:05--3:55, Saturday, Room 252 \\
\Title{Macroscopic dimension and essential manifolds} \Speaker{ Alexander Dranishnikov} \Affil{University of Florida}
\Abstract{
 To learn about a space study its universal cover. This slogan was
 widely applied by Gromov in particular to studying manifolds
of positive scalar curvature (PSC). One of the tools suggested by him is 
the macroscopic dimension $dim_{mc}$. He conjectured that
for $n$-manifold $M$ with PSC the macroscopic dimension of its universal 
cover $\tilde M$ is at most $n-2$. Thus, the question 
for which $n$-manifolds the equality $dim_{mc}\tilde M=n$ holds true is 
of great importance. By the definition of macroscopic 
dimension the equality $dim_{mc}\tilde M=n$ is a form of essentiality 
of $M$ expressed in terms of $\tilde M$. We recall
that a manifold $M$ is essential if the image of the fundamental 
class is nontrivial, $f_*([M])\ne 0$, for a map $f:M\to B\pi$ 
classifying the universal cover of $M$. Gromov conjectured that
for all rationally essential $n$-manifolds there is the equality
$dim_{mc}\tilde M=n$. We will discuss a recent progress on both
Gromov's conjectures. 
}


\hspace*{-4em} 9:00--9:50, Sunday, Room: 252 \\
\Title{The Feichtinger conjecture for group representation frames}\Speaker{ Deguang Han}\Affil{ University of Central Florida} \Abstract{ 
  The Feichtinger frame conjecture states that every bounded
(from below) frame is a finite union of basic Riesz sequences. This
conjecture turns out to be equivalent to the Kadison-Singer pure state
extension problem and several other well-known unsettled problems.  In
this talk I will focus on a few aspects of this conjecture related to
group representation frames and exponential frames for fractal measures.
Additionally I will also briefly discuss its connection with a new duality
principle for group representations and the II$_1$ factor classification
problem
.  }



\hspace*{-4em} 10:00--10:50, Sunday, Room 252 \\
\Title{ Subgroup distortion and bounded cohomology} \Speaker{Indira Chatterji} \Affil{ University State University} \Abstract{
For a connected Lie group G, we show in a joint work with Mislin, Pittet and
Saloff-Coste that the following three conditions are equivalent:
1. The fundamental group of G embeds quasi-isometrically in the universal
cover of G.
2. All Borel (i.e. measurable) cohomology classes of G with integer
coefficients are bounded.
3. The radical of G is linear.

In this talk I will give some motivations for this result, introduce all the
objects used in the statement and explain on an example some ideas used in
the proof.
}

\hspace*{-4em} 11:00--11:50, Sunday, Room: 252 \\

\Title{ Von Neumann algebras with unique group measure space Cartan subalgebra} \Speaker{Ionut Chifan} \Affil{Vanderbilt University } 
\Abstract{
In this talk I will introduce a class of groups $\mathcal {CR}$ satisfying the following property: \vskip 0.05in \noindent \emph{Any} free,
ergodic, measure preserving action on a probability space of \emph{any} group $\Gamma\in \mathcal {CR}$ gives rise to a von Neumann algebras
with unique group measure space Cartan subalgebra. \vskip 0.05in \noindent I will also discuss some applications of this result to
$W^*$-superrigidity. This is joint work with Jesse Peterson.
}



\newpage
\begin{center}
{\DateFont Contributed Talks}
\end{center}
\bigskip
\hspace*{-4em} 11:35--12:00, Saturday, Room: 252\\
\Title{Automatic continuity of orthogonality preserving linear maps
} \Speaker{Timur Oikhbeg}
\Affil{ University of California at Irvine}
 \Abstract{ 
Several known results assert that an orthogonality (resp. disjointness) preserving bijection between C*-algebras (function spaces) must be continuous. In this work, we establish the automatic continuity of bijections from a C*-algebra to a Banach space, provided the images of orthogonal elements satisfy a certain geometric orthogonality condition. Related results for vector-valued spaces of continuous functions are also obtained. This is joint work with A. M. Peralta and M. Ramirez. }

\hspace*{-4em} 11:35--12:00, Saturday, Room: 274\\
\Title{Criteria for irregularity} \Speaker{Gabriel Prajitura}
\Affil{SUNY-Brockport}
 \Abstract{ 
We will discuss sufficient conditions for the existence of irregular vectors for Banach space operators.
}



\hspace*{-4em} 4:10--4:35, Saturday, Room: 252\\
\Title{On the classification problem for a class of crossed product $C^*$-algebras} \Speaker{Jos\'e Carri\'on}
\Affil{University} \Abstract{A (discrete) residually finite group $G$ acts on a profinite
completion $\tilde{G}$ by left translation---a basic example is the
action of $\mathbb{Z}$ on the $p$-adic integers $\mathbb{Z}_p$.  We
study the classification of the corresponding crossed product
$C^*$-algebra $C(\tilde{G})\rtimes G$ via $K$-theoretical invariants.

The eponymous $C^*$-algebras studied by Bunce and Deddens in the 1970s
may be regarded as the case $G = \mathbb{Z}$ and, in analogy with this
case, the so-called generalized Bunce-Deddens algebra
$C(\tilde{G})\rtimes G$ was shown by Orfanos to be simple, separable
and nuclear, and to have real rank zero and stable rank one.  We show
that for a large class of groups (which includes the discrete
Heisenberg group, for example) the corresponding generalized
Bunce-Deddens algebra is classified by its Elliott invariant.
 }


 
\hspace*{-4em} 4:10--4:35, Saturday, Room: 274\\
\Title{Carleson measures, Riemann-Stieltjes and multiplication operators on $F(p,q,s)$ spaces} \Speaker{Ruhan Zhao} \Affil{SUNY-Brockport } \Abstract{ Let T be a nonnegative Borel measure on the unit disk of complex plane. We characterize those measures $T$ such that the general family of spaces of analytic functions, $F(p, q, s)$, which contains many classical function spaces, including the Bloch space, BMOA and the $Q_s$ spaces, are embedded boundedly or compactly into a tent-type spaces. The results are applied to characterize boundedness and compactness of Riemann-Stieltjes operators and multiplication operators on $F(p, q, s)$.
}
\newpage
\hspace*{-4em} 4:45--5:10, Saturday, Room:  252\\
\Title{Nontrivially Noetherian and Artinian C*-algebras} \Speaker{Taylor Hines} \Affil{Purdue University} \Abstract{We say that a C*-algebra is Noetherian if it satisfies the ascending
chain condition for two-sided closed ideals.  A nontrivially
Noetherian C*-algebra is one with infinitely many ideals.  Here, we
show that many nontrivially Noetherian C*-algebras exist, and that a
separable C*-algebra is Noetherian if and only if it contains
countably many ideals and has no infinite strictly ascending chain of
primitive ideals.  Furthermore, we prove that every Noetherian
C*-algebra has a finite-dimensional center.  Where possible, we extend
results about the ideal structure of C*-algebras to Artinian
C*-algebras (those satisfying the descending chain condition for
closed ideals).
 }

 
 \hspace*{-4em} 4:45--5:10, Saturday, Room: 274\\
\Title{Dirac operators on the quantum punctured disk} \Speaker{Matt McBride}
\Affil{IUPUI} \Abstract{ I study quantum analogs of the Dirac type operator $-2\overline{z}\frac{\partial}{\partial\overline{z}}$ on the punctured disk, subject to the Atiyah-Patodi-Singer boundary conditions.  I construct a parametrix of the quantum operator and show that it is bounded outside of the zero mode. 
annulus. }


\hspace*{-4em}
 5:20--5:45, Saturday, Room: 252\\
\Title{"Good" semigroups and their applications to fourier multipliers}
\Affil{University of Illinois at Urbana-Champaign } \Abstract{We shall define what it means for a semigroup of completely
positive, self-adjoint operators on a von Neumann algebra to be "good" in
the sense of Varopoulos. Using a factorization trick of Marius Junge, we
shall show that certain semigroups on group von Neumann algebras are
"good." Finally, we shall discuss some consequences of this fact to
boundedness of fourier multipliers on noncommutative $L_p$ spaces. This is
joint work with Marius Junge and Tao Mei.
 }

%\hspace*{-4em}
% 5:20--5:45, Saturday, Room: 274\\

\end{document}
\newpage
\hspace*{-4em}
 12:00--12:25, Sunday, Room: 252\\
\Title{Why Random Groups have Strong Mostow Rigidity} \Speaker{Paul Schupp}
\Affil{ University of Illinois at
Urbana-Champaign } 
\Abstract{     Rigidity is  pervasive in hyperbolic geometry.  A striking example is that
(Angle, Angle, Angle) is a congruence in standard plane hyperbolic geometry:  The
measures of the angles of a triangle completely determine everything about the 
triangle.  A very deep aspect of hyperbolic rigidity is the Mostow Rigidity Theorem:


{\bf Theorem.} {\it If $X$ and $Y$ are two complete, connected, finite volume hyperbolic manifolds 
of dimension $d \ge 3$ then $X$ and $Y$ are isometric if and only if their fundamental
groups $\Pi_1 (X)$ and $\Pi_1 (Y)$ are isomorphic.}   


    In other words,  hyperbolic isometry is completely determined by the associated groups,
ie, the fundamental groups of the two spaces.


     Let $G = \langle x_1, ..., x_k; r  \rangle$ be a ``random one-relator group'', that is,
the defining relator is a  long random word on the group alphabet $\Sigma = { x_1 , ..., x_k}^{\pm 1}$.
It is now well-known that with probability $1$ the group $G$ is Gromov hyperbolic and thus
the associated geometric space, the Cayley graph $\Gamma(G)$ for the given presentation, is a
hyperbolic metric space.  Now let $H = \langle x_1,..,x_k; s \rangle$ be another random
group on the same set of generators but with a  random relator $s$.  The question is ``How
can $H$ be isomorphic to $G$.''


Kapovich, Schupp and Shpilrain $[1]$ prove:

{\bf Theorem.}
 {\it With probability $1$, $G$ and $H$ are algebraically isomorphic if and only if
their associated Cayley graphs $\Gamma(G)$ and $\Gamma(H)$ are isomorphic as labelled graphs
by a graph isomorphism which is only allowed to permute the label set ${x_1,..,x_k}^{\pm 1}$}


    A group $G$ is \emph{complete} if $G$ has trivial center and trivial outer automorphism group.
Thus $G$ is cannonically isomorphic to its automorphism group $Aut(G)$.  No specific examples of
a nontrivial one-relator group are known  but Kapovich, Schupp and Shpilrain $[1]$ prove 
     

{\bf Theorem.}  {\it With probability $1$, a random one-relator group $G$ is a complete group.}


   Kolmogorov complexity is a general theory of ``descriptive complexity''.  The basic idea
is that a long random word $r$ is its one shortest description up to linear compression.
For any finite group presentation $\Pi = \langle y_1,...,y_p, s_1,...,s_m \rangle$,
define the length $\it{l}_1 (\Pi)$ to be the sum of the lengths of all the relators $s_j$.
Kapovich and Schupp$[2]$ prove that the length of an arbitrary presentation of $G$
cannot be too much shorter than the length $|r|$ of the given defining relator. Thus a
presentation given by a long random relator is ``essentially incompressible''.}
\end{document}
 Giorgi Shonia
Ohio University Lancaster

Kelly Bickel
Graduate Student, Mathematics Department
Washington University in St. Louis
St. Louis, MO 63130
