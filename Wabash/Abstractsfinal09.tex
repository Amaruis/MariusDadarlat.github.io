\documentclass[11pt,leqno]{article}
\usepackage{amscd}
\usepackage{amssymb}
\usepackage{amsthm}

%\documentstyle[11pt,fleqn]{article}
\addtolength{\oddsidemargin}{-4em}
\addtolength{\textwidth}{8em} \addtolength{\topmargin}{-10ex}
\addtolength{\textheight}{30ex}
\newfont{\eu}{eufm10 scaled\magstephalf}
\font\TitleFont=cmbx10 at 20.74truept
\font\DateFont=cmbx10 at 14.40truept
\font\SubTitleFont=cmdunh10 at 12pt
\newcommand{\Title}[1]{\noindent {\bf #1}\\}
\newcommand{\Speaker}[1]{\hspace*{2em} #1,\ }
\newcommand{\Affil}[1]{\noindent{\em #1}\\}
\newcommand{\Abstract}[1]{ #1 \vspace{.3in}}
\newcommand{\st}{^{\textstyle{\ast}}}
\newcommand{\li}{\langle}
\newcommand{\ri}{\rangle}
\newcommand{\vp}{\nu}
\newcommand{\rmA}{\ \ \ \ LE 101}
\newcommand{\rmB}{\ \ \ \ LE 102}
\newcommand{\rmC}{\ \ \ \ LE 103}
\newcommand{\rmD}{\ \ \ \ LE 104}
\pagestyle{empty}
\title{Test format}
\author{Marius Dadarlat}
\begin{document}
\begin{center}
\bigskip
{\TitleFont WABASH}\\
\smallskip
{\SubTitleFont EXTRAMURAL MODERN ANALYSIS}\\
\smallskip
{\TitleFont MINICONFERENCE}\\
\vspace{.2in} {\DateFont   October 3 and 4, 2009}\\
\vspace{.2in}
{\TitleFont Abstracts}\\
\vspace{.4in}
{\DateFont Invited Talks}\\
\end{center}
\bigskip

\hspace*{-4em} 9:30--10:20, Saturday, Room: 252 \\
\Title{Connes-Chern character and eta cocycles} \Speaker{Henri Moscovici} \Affil{Ohio State University } \Abstract{
An intriguing avatar of Connes' dual Chern character in K-homology
assumes the form of a higher eta cocycle.
After briefly recounting the appearance of these cocycles
in the work of Connes and myself,  also of Getzler and Wu,
we shall talk about our current joint work with M. Lesch and
M. Pflaum. In it, such eta cochains arise naturally in the
process of producing concrete cocycle realizations of the
Connes-Chern character of a Dirac operator on a manifold
with boundary in terms of relative cyclic cohomology.
In turn, these explicit relative cocycles, which also
remember the boundary data, give rise to topological pairing
formulas with remarkable geometric consequences.}


\hspace*{-4em}
10:30--11:20, Saturday, Room: 252 \\
\Title{Operator norm localization and its applications to K-theory } \Speaker{Guoliang Yu} \Affil{ Vanderbilt University } \Abstract{We introduce a geometric property which will allow us to estimate the operator norm in a local way
and discuss its applications to operator K-theory.  This is partly joint work with Erik Guentner and Romain Tessera.}


\hspace*{-4em} 2:00--2:50, Saturday, Room: 252 \\
\Title{C*-metric spaces and the distances between them} \Speaker{Marc Rieffel}
\Affil{  University of California at Berkeley} \Abstract{I will describe my present understanding of what one
should mean by a compact C*-metric space, and of ways by which
one can define the distances between them, in analogy with
Gromov-Hausdorff distance between ordinary compact metric
spaces. I will give some examples, notably those associated with
statements in the literature of high-energy physics that "matrix
algebras converge to the sphere". }


\hspace*{-4em} 3:05--3:55, Saturday, Room 252 \\
\Title{Multilinear processes with fractional rank} \Speaker{ Ciprian Demeter} \Affil{Indiana University}
\Abstract{
We investigate averages from Additive Combinatorics (relevant to detecting patterns, such as arithmetic progressions), and Harmonic Analysis, which exhibit fractional rank. A relation between rank and dimension is established, that guarantees that a given process has Fourier complexity. The argument is a mixture of Combinatorics and multi-scale Analysis. 
}


\hspace*{-4em} 9:00--9:50, Sunday, Room: 252 \\
\Title{Quantum relations}\Speaker{ Nik Weaver}\Affil{Washington University } \Abstract{ 

A relation on a set $X$ is a subset of $X^2$.  We define a quantum
relation on a von Neumann algebra M to be a weak* closed operator
bimodule over the commutant $M'$.  This is not the obvious definition
but we claim it is the "right" definition.  It effectively reduces
to the classical notion in the atomic abelian case.  Despite
appearances our definition is effectively representation independent,
and indeed it has an elegant intrinsic characterization.

Various classical structures defined in terms of relations (equivalence
relations, graphs, partial orders, metrics, uniformities) now have
natural quantum analogs and our general treatment of quantum relations
covers all of these cases.  For example, quantum preorders are just
weak* closed unital operator algebras, and basic results in the
theory of these algebras (e.g., reflexivity of commutative subspace
lattices) appear as special cases of our general theory.  This is joint
work with Greg Kuperberg.  }



\hspace*{-4em} 10:00--10:50, Sunday, Room 252 \\
\Title{ K-theory for subspaces of groups} \Speaker{Jacek Brodzki} \Affil{ University of Southampton} \Abstract{
Any discrete metric space $S$ with a sufficient amount of partially defined symmetries can be equipped with a very natural C*-algebra $C^*(X)$ which can be regarded as an analogue of the reduced C*-algebra of a group. A very interesting class of examples of this phenomenon is provided by metric subspaces of discrete groups, and we shall explore this in some detail. A natural question: "When does an embedding of a metric space $X$ into a discrete group $G$ induce a $C^*$-algebra homomorphism $C^*(X) \rightarrow
C^*_r(G)$?" leads to a very interesting $C^*$-algebra extension, which unifies several known classical constructions, including the Toeplitz extension, the Cuntz extension, etc. We shall demonstrate how this result can be used to compute the $K$-theory of $C^*$-algebras of this type. }

\hspace*{-4em} 11:00--11:50, Sunday, Room: 252 \\

\Title{The Effros-Hahn Conjecture for Dynamical Systems and Groupoids} \Speaker{Marius Ionescu} \Affil{ University of  Connecticut} 
\Abstract{ A dynamical system 
$(A,G,\alpha)$, where $A$ is a $C^*$-algebra,
$G$ is a locally compact group and $\alpha$ is a strongly continuous
homomorphism of $G$ into $\mathrm{Aut}A$, is called \emph{EH-regular}
if every primitive ideal of the crossed product $A\rtimes_{\alpha}G$
is induced from a stability group. In their 1967 Memoir, Effros and
Hahn conjectured that if $(G,X)$ was a second countable locally compact
transformation group with $G$ amenable, then 
$\bigl(C_{0}(X),G,\mathrm{lt}\bigr)$
should be EH-regular. This conjecture, and its generalization to dynamical
systems, was proved by Gootman and Rosenberg building on results due
to Sauvageot. In this talk, which is based on joint work with Dana
P. Williams, I will present some recent results regarding extensions
of the Effros-Hahn conjectures to groupoid $C^*$-algebras.}



\newpage
\begin{center}
{\DateFont Contributed Talks}
\end{center}
\bigskip
\hspace*{-4em} 11:35--12:00, Saturday, Room: 252\\
\Title{Tilings, and Baum-Connes Conjecture} \Speaker{Semail Ulgen Yildirim}
\Affil{Northwestern University}
 \Abstract{We recall the work on crossed product $C^*$-algebras such as the $C^*$-algebra $A = C(\Omega) \rtimes R^d$ where $R^d$  acts on $C(\Omega)$ by translations and the hull $\Omega$ is a compact space formed by translations of a given tiling $T$. J. Bellissard defined the notion of a hull $(\Omega, R^d, T)$ to model aperiodic solids. The hull is a dynamical system with group $R^d$ acting by homeomorphisms on a compact metrizable space. In the case of a perfect crystal, with translation group $G$, the hull is
homeomorphic to the $d-dimensional$ torus $T^d$. With any dynamical system, there is a canonical $C^*$-algebra, namely the crossed product $C^*$-algebra  $A = C(\Omega) \rtimes R^d$. We modify this algebra  by enlarging the hull
after including rotational symmetry in addition to translational symmetry on tiles, in particular on aperiodic tilings and call it the modified Bellissard Algebra. In the periodic case one can study the $K$-theory of this modified  $C^*$-algebra and try to detect the
type of the crystal. We briefly recall Baum-Connes Conjecture and mention the use of the
proven results of this famous Baum Connes Conjecture. }

\hspace*{-4em} 11:35--12:00, Saturday, Room: 274\\
\Title{Linear dynamics in nonseparable spaces} \Speaker{Gabriel Prajitura}
\Affil{SUNY-Brockport}
 \Abstract{  We will discuss ideas of generalizing concepts as hypercyclicity to the case of nonseparable spaces.
 }

%\hspace*{-4em}
%3:10--3:30, Saturday, Room: 274\\
%\Title{ Is stability a stable property? } \Speaker{Dan Kucerovsky}
%\Affil{University of New Brunswick} \Abstract{A C*-algebra B is said to be
%stable if $B\otimes K$ is isomorphic to B, where $K$ is the canonical compact
%operators on a separable Hilbert space. Stability is an interesting and
%important property of C*-algebras. One would hope that if the algebras on the
%ends of a short exact sequence are stable, then the one in the middle is.
%Similarly, if a matrix algebra $M_2 (B)$ is stable, one would hope that B is
%stable.
%
%Both of these properties are not true in general. We show that there is a
%simple condition, called the corona factorization condition, that can be
%imposed on the algebra B to make both of the above properties true. The corona
%factorization property is motivated by KK-theory. Since KK-theory is not
%sensitive to whether or not an algebra is stable, it is surprising that a
%condition arising from KK-theory would have such a decisive impact on
%stability-related properties of an algebra.}

\hspace*{-4em} 4:10--4:35, Saturday, Room: 252\\
\Title{Fourier multipliers on noncommutative $L^p$ spaces} \Speaker{Tao Mei}
\Affil{University of Illinois at Urbana-Champaign} \Abstract{
Abstract: I will introduce our recent research on boundedness of Fourier multipliers on noncommutative $L^p$ spaces.  As applications, we find new examples of quantum metric spaces. The talk is based on joint works with M. Junge. }


 
\hspace*{-4em} 4:10--4:35, Saturday, Room: 274\\
\Title{Essential norms of composition operators between Bloch type spaces} \Speaker{Ruhan Zhao} \Affil{SUNY-Brockport } \Abstract{For $\alpha>0$, the $\alpha$-Bloch space is the space of all analytic
functions $f$ on the unit disk $D$ satisfying
$$
\|f\|_{B^{\alpha}}=\sup_{z\in D}|f'(z)|(1-|z|^2)^{\alpha}<\infty.
$$
Let $\varphi$ be an analytic self-map of $D$. 
We show that, for $0<\alpha,\beta<\infty$,
the essential norm of the composition operator $C_{\varphi}$ 
mapping from $B^{\alpha}$ to $B^{\beta}$ can be given by the following formula:
$$
\|C_{\varphi}\|_e=\left(\frac{e}{2\alpha}\right)^{\alpha}\limsup_{n\to\infty}
n^{\alpha-1}\|\varphi^n\|_{B^{\beta}}.
$$}

\hspace*{-4em} 4:45--5:10, Saturday, Room:  252\\
 \Title{$p$-operator spaces and approximation properties.} \Speaker{Jung-Jin Lee}
\Affil{University of Illinois at Urbana-Champaign } \Abstract{Let $G$ be a discrete group. Haagerup showed that $G$ is weakly amenable if and only if $C^*_{\lambda}(G)$ has the completely bounded approximation property. Haagerup and Kraus later considered a slightly weaker property, say the approximation property, and showed that $G$ has the approximation property if and only if $C^*_{\lambda}(G)$ has the operator approximation property. We extend these results using $p$-operator space, a generalization of operator spaces modeled on $L_p$ spaces. This is a joint work with Guimei An and Zhong-Jin Ruan. }

 \hspace*{-4em} 4:45--5:10, Saturday, Room: 274\\
\Title{Compositional Disjoint Hypercyclicity Equals Disjoint Supercyclicity} \Speaker{Ozgur Martin} \Affil {Bowling Green State University} \Abstract{ We say a sequence of continuous operators $\{T_n: n \geq 0\}$ on a topological vector space $X$ is \textbf{hypercyclic} ({\it resp.} \textbf{supercylic}) if there exists an element $x$ in $X$ such that the set $\{T_n(x): n \geq 0\}$ ({\it resp.} the projective orbit $\{ \lambda T_n(x): n \geq 0, \lambda \in \mathbb{C} \}$) is dense in $X$. When $X$ is the set of holomorphic functions on a simply connected domain, Bernal, Bonilla, and Calder\'on showed that for most sequences of  composition operators induced by automorphic symbols  the notions of hypercyclicity and supercyclicity  coincide. In this talk, we will show that  this result holds for all sequences and also can be generalized to the notion of disjointness in hypercyclicity introduced by Bernal and also independently by B\`es and Peris. This is a joint work with Juan P. B\`es (Bowling Green State University).}

\hspace*{-4em}
 5:20--5:45, Saturday, Room: 252\\
\Title{A Hausdorff-Young Inequality for Locally Compact Quantum Groups} \Speaker{Tom Cooney} \Affil{University of Illinois at
Urbana-Champaign } \Abstract{The classical Hausdorff-Young inequality was extended to
unimodular locally compact groups by Kunze and to the general case of a
locally compact group $G$ by Terp. This was done by considering the group
von Neumann algebra $L(G)$ as the object dual to $L^\infty(G)$ and using 
non-commutative $L^p$ spaces. We extend this to the locally compact
quantum group case. }

\hspace*{-4em}
 5:20--5:45, Saturday, Room: 274\\
\Title{D-bar Operators on Quantum Domains} \Speaker{Matt McBride}
\Affil{University of Illinois at Urbana-Champaign} \Abstract{We study the index problem for the d-bar operators subject to
Atiyah-Patodi-Singer boundary conditions on noncommutative disk and
annulus. }

\hspace*{-4em}
 5:55--6:20, Saturday, Room: 252\\
\Title{Complexes of Groups, the Isocohomological property, and Rapid Decay} \Speaker{Bobby Ramsey}
\Affil{IUPUI} \Abstract{The isocohomological property for a discrete group is the statement that
for all coefficient modules, every cocycle has a representative of
polynomial growth.  This property is known for a wide class of groups,
and was central in Connes' and Moscovici's approach to proving the
strong Novikov conjecture for hyperbolic groups. 
We introduce complexes of groups and examine the isocohomological and 
Rapid Decay properties for their fundamental groups.  
}

\newpage
\hspace*{-4em}
 12:00--12:25, Sunday, Room: 252\\
\Title{Why Random Groups have Strong Mostow Rigidity} \Speaker{Paul Schupp}
\Affil{ University of Illinois at
Urbana-Champaign } 
\Abstract{     Rigidity is  pervasive in hyperbolic geometry.  A striking example is that
(Angle, Angle, Angle) is a congruence in standard plane hyperbolic geometry:  The
measures of the angles of a triangle completely determine everything about the 
triangle.  A very deep aspect of hyperbolic rigidity is the Mostow Rigidity Theorem:


{\bf Theorem.} {\it If $X$ and $Y$ are two complete, connected, finite volume hyperbolic manifolds 
of dimension $d \ge 3$ then $X$ and $Y$ are isometric if and only if their fundamental
groups $\Pi_1 (X)$ and $\Pi_1 (Y)$ are isomorphic.}   


    In other words,  hyperbolic isometry is completely determined by the associated groups,
ie, the fundamental groups of the two spaces.


     Let $G = \langle x_1, ..., x_k; r  \rangle$ be a ``random one-relator group'', that is,
the defining relator is a  long random word on the group alphabet $\Sigma = { x_1 , ..., x_k}^{\pm 1}$.
It is now well-known that with probability $1$ the group $G$ is Gromov hyperbolic and thus
the associated geometric space, the Cayley graph $\Gamma(G)$ for the given presentation, is a
hyperbolic metric space.  Now let $H = \langle x_1,..,x_k; s \rangle$ be another random
group on the same set of generators but with a  random relator $s$.  The question is ``How
can $H$ be isomorphic to $G$.''


Kapovich, Schupp and Shpilrain $[1]$ prove:

{\bf Theorem.}
 {\it With probability $1$, $G$ and $H$ are algebraically isomorphic if and only if
their associated Cayley graphs $\Gamma(G)$ and $\Gamma(H)$ are isomorphic as labelled graphs
by a graph isomorphism which is only allowed to permute the label set ${x_1,..,x_k}^{\pm 1}$}


    A group $G$ is \emph{complete} if $G$ has trivial center and trivial outer automorphism group.
Thus $G$ is cannonically isomorphic to its automorphism group $Aut(G)$.  No specific examples of
a nontrivial one-relator group are known  but Kapovich, Schupp and Shpilrain $[1]$ prove 
     

{\bf Theorem.}  {\it With probability $1$, a random one-relator group $G$ is a complete group.}


   Kolmogorov complexity is a general theory of ``descriptive complexity''.  The basic idea
is that a long random word $r$ is its one shortest description up to linear compression.
For any finite group presentation $\Pi = \langle y_1,...,y_p, s_1,...,s_m \rangle$,
define the length $\it{l}_1 (\Pi)$ to be the sum of the lengths of all the relators $s_j$.
Kapovich and Schupp$[2]$ prove that the length of an arbitrary presentation of $G$
cannot be too much shorter than the length $|r|$ of the given defining relator. Thus a
presentation given by a long random relator is ``essentially incompressible''.}
\end{document}
 