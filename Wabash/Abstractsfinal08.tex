\documentclass[11pt,leqno]{article}
\usepackage{amscd}
\usepackage{amssymb}

%\documentstyle[11pt,fleqn]{article}
\addtolength{\oddsidemargin}{-4em}
\addtolength{\textwidth}{8em} \addtolength{\topmargin}{-10ex}
\addtolength{\textheight}{30ex}
\newfont{\eu}{eufm10 scaled\magstephalf}
\font\TitleFont=cmbx10 at 20.74truept
\font\DateFont=cmbx10 at 14.40truept
\font\SubTitleFont=cmdunh10 at 12pt
\newcommand{\Title}[1]{\noindent {\bf #1}\\}
\newcommand{\Speaker}[1]{\hspace*{2em} #1,\ }
\newcommand{\Affil}[1]{\noindent{\em #1}\\}
\newcommand{\Abstract}[1]{ #1 \vspace{.3in}}
\newcommand{\st}{^{\textstyle{\ast}}}
\newcommand{\li}{\langle}
\newcommand{\ri}{\rangle}
\newcommand{\vp}{\nu}
\newcommand{\rmA}{\ \ \ \ LE 101}
\newcommand{\rmB}{\ \ \ \ LE 102}
\newcommand{\rmC}{\ \ \ \ LE 103}
\newcommand{\rmD}{\ \ \ \ LE 104}
\pagestyle{empty}
\title{Test format}
\author{Marius Dadarlat}
\begin{document}
\begin{center}
\bigskip
{\TitleFont WABASH}\\
\smallskip
{\SubTitleFont EXTRAMURAL MODERN ANALYSIS}\\
\smallskip
{\TitleFont MINICONFERENCE}\\
\vspace{.2in} {\DateFont   September 6 and 7, 2008}\\
\vspace{.2in}
{\TitleFont Abstracts}\\
\vspace{.4in}
{\DateFont Invited Talks}\\
\end{center}
\bigskip

\hspace*{-4em} 9:30--10:20, Saturday, Room: 252 \\
\Title{Equivariant correspondences and applications} \Speaker{Heath Emerson}
\Affil{  University of Victoria} \Abstract{We develop a purely topological model for Kasparov's equivariant
bivariant K-theory (for locally compact spaces); in other words, the  morphisms
$X \to Y$
in this model are given by purely topological data. As an application, we compute
Lefschetz invariants of arbitrary equivariant Kasparov morphisms $X \to X$ when
X is a smooth manifold carrying a smooth and proper action of a discrete
group. This leads to a generalisation of the classical
Lefschetz
fixed-point formula. }

\hspace*{-4em}
10:30--11:20, Saturday, Room: 252 \\
\Title{Matrix Convexity, Quantum Inequalities,
and Non-commutative Differentiation} \Speaker{Edward G. Effros} \Affil{ University of
California at Los Angeles} \Abstract{
Convexity methods play a central role in both functional analysis and
thermodynamics. This has remained the case in both non-commutative
functional analysis and quantum thermodynamics. The techniques of the
latter subject have required considerable analytic ingenuity, as will be apparent
to anyone interested in non-commutative entropy theory.
We show that some of these difficulties can be eliminated if one uses
simple notions of matrix convexity theory. In particular it is possible to
give “one-line” proofs of some celebrated inequalties due to Lieb and others.
On the other hand, matrix convexity techniques can be used to formulate
notions of smoothness for operator spaces, and more general systems.}

\hspace*{-4em} 2:00--2:50, Saturday, Room: 252 \\
\Title{Dimension(s) for C*-algebras} \Speaker{Andrew Toms} \Affil{ York University } \Abstract{The study of C*-algebras may be regarded as noncommutative topology.  As such, is it natural to ask for C*-algebra generalisations of notions from the study of locally compact Hausdorff spaces.  One such notion is the covering dimension of the space.  In this talk I will survey various methods of generalising this concept to C*-algebras, and also variants on these methods which account more effectively for the "matricial" structure of C*-algebras.  We will define dimensions for C*-algebras by topological, operator algebraic, and homological means, and see how they are sensitive to differences between steadily more exotic C*-algebras.  Finally, we will discuss the relationships between these dimension theories, and their bearing on K-theoretic rigidity phenomena.}


\hspace*{-4em} 3:05--3:55, Saturday, Room 252 \\
\Title{ Group cocycles and the ring of affiliated operators} \Speaker{Jesse Peterson} \Affil{Vanderbilt University} \Abstract{I will present some results (joint work with Andreas Thom) on
cocycles from a group into its left regular representation and also into
the ring of affiliated operators of the group von Neumann algebra.  I will
present a strong generalization of a result of L{\"u}ck and Gaboriau which
states that if $\Lambda$ is a finitely generated normal subgroup of a
group $\Gamma$ with $0<\beta_1^{(2)}(\Gamma)<\infty$ then either
$|\Lambda|< \infty$ or $[\Gamma:\Lambda]<\infty$. }

\hspace*{-4em} 9:00--9:50, Sunday, Room: 252 \\
\Title{Twisting and Rieffel's Deformation of Locally Compact Quantum Groups} \Speaker{ Leonid Vainerman}\Affil{University of Caen} \Abstract{ We develop the twisting construction for locally compact quantum groups.
A new feature, in contrast to my previous joint work with M. Enock, is a
non-trivial deformation of the Haar measure. Then we construct Rieffel's
deformation of locally compact quantum groups and show that it is dual to
the twisting. This allows to give new interesting concrete examples of
locally compact quantum groups, in particular, deformations of the
classical
"az+b" groups and of Woronowicz' quantum "az+b" group. (This is joint work with Pierre Fima. }



\hspace*{-4em} 10:00--10:50, Sunday, Room 252 \\
\Title{Beginnings of a theory of nonlinear noncommutative elliptic partial
differential equations } \Speaker{ Jonathan Rosenberg} \Affil{University of Maryland}
\Abstract{The full development of Connes' program of noncommutative geometry
will eventually lead to the study of noncommutative "geometric
elliptic PDE", for the same reasons that geometric elliptic PDE play an
important role in classical differential geometry.  We discuss the
beginnings of a theory for treating such equations, in the simplest
test case of equations involving the Laplacian on the irrational rotation
algebras.
}

\hspace*{-4em} 11:00--11:50, Sunday, Room: 252 \\

\Title{Turbulence, representations, and trace-preserving actions} \Speaker{Hanfeng Li} \Affil{ SUNY Buffalo} \Abstract{ I will give criteria for tubulence in spaces of
 C*-algebra representations, and indicate how this helps to establish
 results of nonclassifiability by countable structures, for group
 actions on a standard atomless probability space and on the hyperfinite
 II$_1$ factor. This is a joint work with David Kerr and Mikael Pichot.}



%\newpage
\vspace{.5in}
\begin{center}
{\DateFont Contributed Talks}

\vskip 1cm \centerline{Parallel Sessions}
\end{center}
\bigskip
\hspace*{-4em} 11:35--12:00, Saturday, Room: 252\\
\Title{A Singular Subfactor That is Not Strongly Singular} \Speaker{Alain Wiggins}
\Affil{Vanderbilt University}
 \Abstract{A subalgebra $B$ of a II$_1$ factor $M$ is singular if every normalizing unitary in $M$ of $B$ actually lives in $B$. Smith and Sinclair defined $B$ to be $\alpha$-strongly singular in $M$ if there is a constant $0<\alpha\leq 1$ such that
\[
\alpha\Vert u-\mathbb{E}_B(u)\Vert_2\leq \Vert\mathbb{E}_B-\mathbb{E}_{uBu^*}\Vert_{\infty,2}
\]
for all unitaries $u\in M$. $B$ is strongly singular if $\alpha$ can be taken to be 1. Employing techniques from planar algebras, we give an example of a finite index subfactor inclusion $N\subseteq M$ such that $N$ is singular in $M$ but not strongly singular. This is joint work with Pinhas Grossman. }

\hspace*{-4em} 11:35--12:00, Saturday, Room: 274\\
\Title{Irregular orbits of operators} \Speaker{Gabriel Prajitura}
\Affil{SUNY-Brockport}
 \Abstract{ We will discuss a geometric property of orbits which is one side of hyperclicity. }

%\hspace*{-4em}
%3:10--3:30, Saturday, Room: 274\\
%\Title{ Is stability a stable property? } \Speaker{Dan Kucerovsky}
%\Affil{University of New Brunswick} \Abstract{A C*-algebra B is said to be
%stable if $B\otimes K$ is isomorphic to B, where $K$ is the canonical compact
%operators on a separable Hilbert space. Stability is an interesting and
%important property of C*-algebras. One would hope that if the algebras on the
%ends of a short exact sequence are stable, then the one in the middle is.
%Similarly, if a matrix algebra $M_2 (B)$ is stable, one would hope that B is
%stable.
%
%Both of these properties are not true in general. We show that there is a
%simple condition, called the corona factorization condition, that can be
%imposed on the algebra B to make both of the above properties true. The corona
%factorization property is motivated by KK-theory. Since KK-theory is not
%sensitive to whether or not an algebra is stable, it is surprising that a
%condition arising from KK-theory would have such a decisive impact on
%stability-related properties of an algebra.}

\hspace*{-4em} 4:10--4:35, Saturday, Room: 252\\
\Title{Planar algebra of the diagonal subfactor} \Speaker{Paramita Das} \Affil{Vanderbilt University } \Abstract{Starting with a finite set $\{ \theta_i \}_{i \in I}$ of
automorphisms of a $II_1$ factor $N$, one can construct the {\it
diagonal subfactor} $N \subset M_I (N)$ where an element $x \in N$ sits
in $M_I (N)$ diagonally where the $i$-th diagonal element is given by
$\theta_i (x)$.
Diagonal subfactors are among the most basic subfactors. For
instance, a correspondence between amenability of such subfactors and
amenability of the group generated by the automorphisms in $Out(N)$
was obtained by Popa and the standard invariant is well-known.
We will describe the planar algebra of these subfactors.
The action of the planar tangles depends on the cocycle
obstruction for the group generated by $\theta_i$ in $Out(N)$
to have an action. When the obstruction is trivial,
this planar algebra matches with Jones's example of planar
algebra associated to finitely generated group.
This is a joint work with Dietmar Bisch and Shamindra Ghosh. }
\vskip 16pt
 
\hspace*{-4em} 4:10--4:35, Saturday, Room: 274\\
\Title{Characterizations of Bloch type spaces by divided differences} \Speaker{Ruhan Zhao} \Affil{SUNY-Brockport } \Abstract{Bloch type spaces on the unit disk are characterized by using Newton's divided differences. Several other characterizations of Bloch type spaces involving two or more points as well as applications to polynomial interpolation to Bloch type functions are also given.}
\vskip 16pt


\hspace*{-4em} 4:45--5:10, Saturday, Room:  252\\
 \Title{Planar algebra of group-type subfactors} \Speaker{Shamindra Kumar Ghosh}
\Affil{Vanderbilt University } \Abstract{We describe the planar algebra, or equivalently, the standard invariant, of the subfactor $P^H \subset P \rtimes K$ arising from outer actions of two finite groups $H$ and $K$ on a $II_1$-factor $P$. These subfactors, introduced by Bisch and Haagerup, play an important role in the theory providing a very simple mechanism to construct irreducible subfactors whose standard invariant has infinite depth. The planar algebra heavily depends on the cocycle arising as an obstruction to lifting the subgroup $G$ in $Out(P)$ generated by $H$ and $K$. If we assume that the group generated by $H$ and $K$ in $Aut(P)$ intersects trivially with $Inn(P)$, (equivalently, the obstruction is trivial), then the planar algebra has an interesting similarity with IRF models in Statistical Mechanics. This is a joint work with Dietmar Bisch and Paramita Das.  }

\hspace*{-4em} 4:45--5:10, Saturday, Room: 274\\
\Title{A Connection Between Spectral Synthesis, Operator Topologies,\\ Polynomial Interpolation, and Linear Independence of Exponential Series} \Speaker{Ian Nathaniel Deters} \Affil {Bowling Green State University} \Abstract{Diagonal Operators operators on Hilbert spaces have been well studied.  However, one could look at linear operators on the space of functions analytic on a disk of arbitrary radius whose eigenvectors are the monomials. These are interesting operators to consider since they are not defined on a Banach space. Hence, not all of the classical spectral theory applies. For instance, a continuous operator may have unbounded eigenvalues. However, not all is lost since one has the results of complex analysis at one's disposal.

In this talk we will determine what must happen in order for such operators to admit spectral synthesis. These questions lead nicely to creating connections between spectral synthesis of diagonal operators, operator topologies, polynomial approximation, and linear independence of exponential series.

\vskip 1cm



\hspace*{-4em}
 5:20--5:45, Saturday, Room: 252\\
\Title{Littlewood-Paley inequality for operator- valued
function at the end point} \Speaker{Tao Mei}
\Affil{University of Illinois at Urbana-Champaign} \Abstract{We consider $n$ by $n$ matrix-valued function $f$ for large $n$
and corresponding Littlewood-Paley type square functions $G(f)$. We estimate
the $L^p$  boundedness of $G(f)$ at the end point ($p=1,\infty$). We will show and the difference from the scalar-valued
case connections to noncommutative martingales. This is a recent joint work with Javier Parcet.}

\vskip 1cm
\hspace*{-4em}
 5:55--6:20, Saturday, Room: 252\\
\Title{The Isocohomological Property, Higher Dehn Functions, \\ and Relatively
Hyperbolic Groups} \Speaker{Bobby Ramsey}
\Affil{IUPUI} \Abstract{The property that the polynomial cohomology with coefficients of a
finitely generated discrete group is canonically isomorphic to the group
cohomology is called the (weak) isocohomological property for the group.
In the case when a group is of type $HF^\infty$, i.e. that has a
classifying space with the homotopy type of a cellular complex with
finitely many cells in each dimension, we show that the isocohomological
property is equivalent to the classifying space satisfying polynomially
bounded higher Dehn functions.  If a group is hyperbolic relative to a
collection of subgroups, each of which is polynomially combable
(respectively $HF^\infty$ and isocohomological), then we show that the
group itself has these respective properties too.  Combining with the
results of Connes-Moscovici and Dru{\c{t}}u-Sapir we conclude that a
group satisfies the Novikov conjecture if it is relatively hyperbolic to
subgroups that are of property RD, of type $HF^\infty$ and
isocohomological.}

\hspace*{-4em}
 12:00--12:25, Sunday, Room: 252\\
\Title{Property T for discrete quantum groups} \Speaker{Pierre Fima}
\Affil{ University of Illinois at
Urbana-Champaign } \Abstract{We give a simple definition a property T for discrete quantum groups and we show that, for I.C.C. discrete quantum groups, this definition is equivalent to the Connes' property T of the associated von Neumann algebra. This allows us to give the first example of property T discrete quantum group which is not a group using the twisting construction.}
}\end{document}
 